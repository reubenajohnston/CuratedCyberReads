%Copyright 2023, Reuben Johnston, www.reubenjohnston.com
%Dependencies
%$ sudo apt-get install texlive texlive-latex-extra
%Usage
%$ pdflatex curated_list_of_infamous_malware_reuben_johnston.tex

\documentclass[a4paper]{article}
\usepackage[a4paper, portrait, margin=0.5in]{geometry}

% copyright
\usepackage[pages=all, color=black, position={current page.south}, placement=bottom, scale=1, opacity=1, vshift=5mm]{background}
\SetBgContents{
	\tt This work is shared under a \href{https://creativecommons.org/licenses/by-nc-sa/4.0/}{CC BY-NC-SA 4.0 license} unless otherwise noted
}

\usepackage{hyperref}
\usepackage[ampersand]{easylist}

\title{Curated List of Infamous Malware}
\author{Reuben Aaron Johnston, Ph.D. \\ email \href{mailto:reub@jhu.edu}{reub@jhu.edu}}
\date{January 2024}

\begin{document}
	\maketitle
	
	In the spirit of learning, I am sharing this curated list of infamous malware with references to associated writings.  Please feel free to reach out with any suggestions for improvement!  
	
	\bigskip\noindent
	
	\noindent\textbf{Morris Worm (1988):}
	\begin{easylist}[itemize]
	& \href{https://0x00sec.org/t/examining-the-morris-worm-source-code-malware-series-0x02/685}{Examining the Morris Worm Source Code - Malware Series - 0x02, by Cromical, July 2016}
	& \href{https://spaf.cerias.purdue.edu/tech-reps/823.pdf}{The Internet Worm Program: An Analysis (CSD-TR-823), by Eugene Spafford, Purdue, 1988}
	\end{easylist}

	\noindent\textbf{Melissa Virus (1999):}
	\begin{easylist}[itemize]
	& \href{https://www.govinfo.gov/content/pkg/GAOREPORTS-T-AIMD-99-146/pdf/GAOREPORTS-T-AIMD-99-146.pdf}{The Melissa Computer Virus Demonstrates Urgent Need for Stronger Protection Over Systems and Sensitive Data, testimony by Keith A. Rhodes, GAO, April 15, 1999}
	\end{easylist}	

	\noindent\textbf{Iloveyou Virus (2000):}
	\begin{easylist}[itemize]
	& \href{https://www.cnn.com/2020/05/01/tech/iloveyou-virus-computer-security-intl-hnk/index.html}{‘I love you’: How a badly-coded computer virus caused billions in damage and exposed vulnerabilities which remain 20 years on, by James Griffiths, CNN Business, May 3, 2020}
	& Analysis of the ILOVEYOU Worm, by Matt Bishop, University of California at Davis, May 9, 2000
	\end{easylist}	

	\noindent\textbf{Code Red Worm (2001):}
	\begin{easylist}[itemize]
	& \href{https://doi.org/10.1145/501317.501328}{The Code Red Worm, by Hal Berghel, Communications of the ACM, Vol. 44, No. 12, Dec. 2001}
	& \href{https://www.sans.org/white-papers/85/}{The Code Red Worm (v1.2e), by John C. Dolak, Security Essentials}
	\end{easylist}

	\noindent\textbf{Slammer Worm (2003):}
	\begin{easylist}[itemize]
	& \href{https://doi.org/10.1109/MSECP.2003.1219056}{Inside the Slammer Worm, Moore et al., IEEE Security and Privacy, Vol. 1, No. 4, July-Aug. 2003}
	\end{easylist}	

	\noindent\textbf{Blaster Worm (2003):}
	\begin{easylist}[itemize]
	& \href{https://doi.org/10.1109/MSP.2005.106}{The Blaster Worm: Then and Now, Bailey et al., IEEE Security and Privacy, Vol. 3, No. 4, July-Aug. 2005}
	\end{easylist}	

	\noindent\textbf{Mydoom Worm (2004):}
	\begin{easylist}[itemize]
	& \href{https://www.newsweek.com/more-doom-131157}{More Doom?, Newsweek, February 2, 2004}
	& \href{https://unit42.paloaltonetworks.com/mydoom-still-active-in-2019/}{MyDoom Still Active in 2019, by Brad Duncan, Unit 42, Palo Alto Networks, July 26, 2019}
	& \href{https://web.archive.org/web/20190122073704/https://www.symantec.com/security-center/writeup/2004-012612-5422-99}{W32.Mydoom.A@mm, Symantec, January 26, 2004}
	\end{easylist}	

	\noindent\textbf{Darkhotel Cyberspy Campaign (2007):}
	\begin{easylist}[itemize]
	& \href{https://www.kaspersky.com/blog/darkhotel-apt/6613}{Darkhotel: a spy campaign in luxury Asian hotels, by Alex Drozhzhin, Kaspersky Daily, November 10, 2014}
	\end{easylist}	
	
	\noindent\textbf{Conficker Worm (2008):}
	\begin{easylist}[itemize]
	& \href{https://doi.ieeecomputersociety.org/10.1109/MC.2009.198}{On the Trail of the Conficker Worm, George Lawton, IEEE Computer, Vol. 42, June 2009}
	\end{easylist}	

	\noindent\textbf{Stuxnet Attack (2010):}
	\begin{easylist}[itemize]
	& \href{https://www.wired.com/2014/11/countdown-to-zero-day-stuxnet}{An Unprecedented Look at Stuxnet, the World's First Digital Weapon, Kim Zetter, Wired, November 3, 2014}
	& \href{https://www.kaspersky.com/blog/stuxnet-victims-zero/6775}{Stuxnet: Victims Zero, Marvin the Robot, Kaspersky Daily, November 18, 2014}
 	& \href{https://web.archive.org/web/20180608052155/http://www.symantec.com/content/en/us/enterprise/media/security_response/whitepapers/w32_stuxnet_dossier.pdf}{W32.Stuxnet Dossier, Falliere et al., Symantec Security Response, February 2011}
  	& \href{https://doi.org/10.1109/MSPEC.2013.6471059}{The Real Story of Stuxnet, David Kushner, IEEE Spectrum, Vol. 50, No. 3, March 2013}
	\end{easylist}	

	\noindent\textbf{Cryptolocker Ransomware (2013):}
	\begin{easylist}[itemize]
	& Cryptolocker, by Carl Saiyed, ISSA Journal, Vol. 14 No. 4, April 2016
	\end{easylist}	

	\noindent\textbf{Petya Ransomware (2016):}
	\begin{easylist}[itemize]
	& \href{https://www.kaspersky.com/blog/petya-ransomware/11715}{Petya ransomware eats your hard drives, by John Snow, Kaspersky Daily, March 30, 2016}
	\end{easylist}	

	\noindent\textbf{Mirai Botnet (2016):}
	\begin{easylist}[itemize]
	& \href{https://www.wired.com/story/mirai-untold-story-three-young-hackers-web-killing-monster}{The Mirai Confessions: Three Young Hackers Who Built a Web-Killing Monster Finally Tell Their Story, by Andy Greenberg, Wired, November 14, 2023}
	& \href{https://www.kaspersky.com/blog/attack-on-dyn-explained/13325}{How to not break the Internet, by Kate Kochetkova, Kaspersky Daily, October 26, 2016}
	\end{easylist}	
 
	\noindent\textbf{New Petya Ransomware (2017):}
	\begin{easylist}[itemize]
	& \href{https://www.kaspersky.com/blog/new-ransomware-epidemics/17314}{New Petya / NotPetya / ExPetr ransomware outbreak, Marvin the Robot, Kaspersky Daily, June 27, 2017}
	& \href{https://securelist.com/schroedingers-petya/78870}{Schroedinger’s Pet(ya), Kaspersky Securelist, June 27, 2017}
	\end{easylist}	

	\noindent\textbf{EternalBlue+WannaCry (2017):}
	\begin{easylist}[itemize]
	& \href{https://www.kaspersky.com/blog/wannacry-ransomware/16518}{WannaCry: Are you safe?, Alex Perekalin, Kaspersky Daily, May 13, 2017}
 	& \href{https://research.checkpoint.com/2017/eternalblue-everything-know}{ETERNALBLUE – EVERYTHING THERE IS TO KNOW, by Nadav Grossman, Check Point Research, September 29, 2017}
  	& \href{https://web.archive.org/web/20220302113527/https://www.microsoft.com/security/blog/2017/06/30/exploring-the-crypt-analysis-of-the-wannacrypt-ransomware-smb-exploit-propagation}{Exploring the crypt: Analysis of the WannaCrypt ransomware SMB exploit propagation, Microsoft Defender Security Research Team, June 30, 2017}
	\end{easylist}	

	\noindent\textbf{Solarwinds Attack (2019):}
	\begin{easylist}[itemize]
	& \href{https://www.wired.com/story/the-untold-story-of-solarwinds-the-boldest-supply-chain-hack-ever}{The Untold Story of the Boldest Supply-Chain Hack Ever, by Kim Zetter, Wired, May 2, 2023}
	& \href{https://www.npr.org/2021/04/16/985439655/a-worst-nightmare-cyberattack-the-untold-story-of-the-solarwinds-hack}{"A 'Worst Nightmare' Cyberattack: The Untold Story Of The SolarWinds Hack", Raston, All Things Considered, NPR}
	\end{easylist}	

	\noindent\textbf{CosmicStrand UEFI rootkits (2020):}
	\begin{easylist}[itemize]
	& \href{https://securelist.com/cosmicstrand-uefi-firmware-rootkit/106973}{CosmicStrand: the discovery of a sophisticated UEFI firmware rootkit, by Great, Kaspersky Securelist, Jul 25, 2022}
	& \href{https://arstechnica.com/information-technology/2022/07/researchers-unpack-unkillable-uefi-rootkit-that-survives-os-reinstalls}{Discovery of new UEFI rootkit exposes an ugly truth: The attacks are invisible to us, by Dan Goodin, July 26, 2022}
	\end{easylist}	

	\noindent\textbf{operation triangulation (2023):}
	\begin{easylist}[itemize]
	& \href{https://securelist.com/operation-triangulation-the-last-hardware-mystery/111669/}{Operation Triangulation: The last (hardware) mystery, by Boris Larin, SecureList, Kaspersky, December 27, 2023}
	\end{easylist}		

	\bigskip\noindent
\end{document}
